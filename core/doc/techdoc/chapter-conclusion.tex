

\chapter{Conclusion}
\label{chap-conclusion}


Model rocketry is an intriguing sport which combines various fields
ranging from aerodynamic design to model construction to
pyrotechnics.  At its best, it works as an inspiration for youngsters
to study engineering and sciences.

This thesis work provides one of the computer-age tools for everybody
intrested in model rocket design.  Providing everybody free access to
a full-fledged rocket simulator allows many more hobbyists to
experiment with different kinds of rocket designs and become more
involved in the sport.  The most enthusiastic rocketeers may dive
even deeper and get to examine not only the simulation results, but
also how those simulations are actually performed.

The software produced contains an easy-to-use interface, which allows
new users to start experimenting with the minimum effort.  The
back-end is designed to be easily extensible, in anticipation of
future enhancements.  This thesis also includes a step-by-step process
for computing the aerodynamical characteristics of a rocket and for
simulating its flight.  These are the current default implementations
used by the software.

Comparison to experimental data shows that the most important
aerodynamical parameters for flight simulation---the center of
pressure location and drag coefficient---are simulated with an
accuracy of approximately 10\% at subsonic velocities.  In this
velocity regime the accuracy of the simulated altitude is on par with
the commercial simulation software RockSim.  While comparison with
supersonic rockets was not possible, it is expected that the
simulation is reasonably accurate to at least Mach~1.5.

The six degree of freedom simulator also allows simulating rocket roll
in order to study the effect of roll stabilization, a feature
not available in other hobby-level rocket simulators.  While the
comparison with experimental data of a rolling rocket was
inconclusive as to its accuracy, it is still expected to give valuable
insight into the effects of roll during flight.

The external listener classes that can be attached to the simulator
allow huge potential for custom extensions.  For example testing the
active roll reduction controller that will be included in the
successor project of Haisun��t� would have been exceedingly difficult
without such support.  By interfacing the actual controller with a
simulated flight environment it was possible to discover various bugs
in the controller software that would otherwise have gone undetected.

Finally, it must be emphasized that the release of the OpenRocket
software is not the end of this project.  In line with the Open Source
philosophy, it is just the beginning of its development cycle,
where anybody with the know-how can contribute to making OpenRocket an
even better simulation environment.



